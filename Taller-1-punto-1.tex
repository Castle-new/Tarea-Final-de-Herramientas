\documentclass[fleqn]{article}

%% Created with wxMaxima 20.06.6

\setlength{\parskip}{\medskipamount}
\setlength{\parindent}{0pt}
\usepackage{iftex}
\ifPDFTeX
  % PDFLaTeX or LaTeX 
  \usepackage[utf8]{inputenc}
  \usepackage[T1]{fontenc}
  \DeclareUnicodeCharacter{00B5}{\ensuremath{\mu}}
\else
  %  XeLaTeX or LuaLaTeX
  \usepackage{fontspec}
\fi
\usepackage{graphicx}
\usepackage{color}
\usepackage{amsmath}
\usepackage{grffile}
\usepackage{ifthen}
\newsavebox{\picturebox}
\newlength{\pictureboxwidth}
\newlength{\pictureboxheight}
\newcommand{\includeimage}[1]{
    \savebox{\picturebox}{\includegraphics{#1}}
    \settoheight{\pictureboxheight}{\usebox{\picturebox}}
    \settowidth{\pictureboxwidth}{\usebox{\picturebox}}
    \ifthenelse{\lengthtest{\pictureboxwidth > .95\linewidth}}
    {
        \includegraphics[width=.95\linewidth,height=.80\textheight,keepaspectratio]{#1}
    }
    {
        \ifthenelse{\lengthtest{\pictureboxheight>.80\textheight}}
        {
            \includegraphics[width=.95\linewidth,height=.80\textheight,keepaspectratio]{#1}
            
        }
        {
            \includegraphics{#1}
        }
    }
}
\newlength{\thislabelwidth}
\DeclareMathOperator{\abs}{abs}
\usepackage{animate} % This package is required because the wxMaxima configuration option
                      % "Export animations to TeX" was enabled when this file was generated.

\definecolor{labelcolor}{RGB}{100,0,0}

\begin{document}

\pagebreak{}
{\Huge {\scshape Ejercicio 3 de la seccion 1.1.6}}
\setcounter{section}{0}
\setcounter{subsection}{0}
\setcounter{figure}{0}

Realice un código en MAXIMA para el cual, dado tres vectores que formen loslados de un triángulo calcule el centroide (Problema 3 Sección 1.1.6).Trate de hacer este problema de forma que pueda reutilizar su código paratres vectores cualesquiera.


\noindent
%%%%%%%%
%% INPUT:
\begin{minipage}[t]{4.000000em}\color{red}\bfseries
 \ensuremath{\longrightarrow}  
\end{minipage}
\begin{minipage}[t]{\textwidth}\color{blue}
f(a,b,c):=(a+b+c)/3;
\end{minipage}
%%%% OUTPUT:
\[\displaystyle \tag{\% o56} 
\operatorname{f}\left( a\operatorname{,}b\operatorname{,}c\right) \operatorname{:=}\frac{a+b+c}{3}\mbox{}
\]
%%%%%%%%%%%%%%%%
La funcion anterrior recibe 3 vectores ya sea en r2 o r3 y me arroja unvector en ese mismo espacio, el cual es el centroide del triangulo que seforma entre a,b,c.


\noindent
%%%%%%%%
%% INPUT:
\begin{minipage}[t]{4.000000em}\color{red}\bfseries
 \ensuremath{\longrightarrow}  
\end{minipage}
\begin{minipage}[t]{\textwidth}\color{blue}
f([2,2],[10,2],[6,8]);
\end{minipage}
%%%% OUTPUT:
\[\displaystyle \tag{\% o57} 
[6\operatorname{,}4]\mbox{}
\]
%%%%%%%%%%%%%%%%
\end{document}
